\pagebreak
\section*{Sortialgorithmen}
\subsection*{Bubble Sort}
Sortiert von Hinten nach Vorne. Immer zwei Elemente werden verglichen. Wenn Element i-1 kleiner i dann wird getauscht. Vorne ist immer sortiert. $O(n^2)$
\begin{minted}{java}
void BubbleSort(int[] a) {
  int n = a.length;
  for (int i = 0; i < n - 1; i++) {
    for (int j = n - 1; j > i; j--) { 
      if (a[j] < a[j - 1]) {
        int t = a[j - 1]; a[j - 1] = a[j]; a[j] = t; 
      }
    }
  }
}

\end{minted}
\subsection*{Insertion Sort}
Vorne nach Hinten. Vorne ist aufsteigend sortiert. In jeder Iteration wird ein Element mehr genommen und geschaut wo es hingehört. $O(n^2)$
Insertion Sort mit binary Search hat $O(n*\log{}n)$  
\begin{minted}{java}
void InsertionSort1(int[] a) { 
  int n = a.length;
  for (int i = 1; i < n; i++) {
    for (int j = i; j > 0 && a[j - 1] > a[j]; j--) {
       int t = a[j - 1]; a[j - 1] = a[j]; a[j] = t; 
    }
  } 
}
\end{minted}
\subsection*{Selection Sort}
Vorne nach Hinten. Vorne ist aufsteigend sortiert. Es wird immer das kleinste Element im Array gesucht. $O(n^2)$ 
\begin{minted}{java}
void SelectionSort(int[] a) {
  int n = a.length;
  for (int i = 0; i < n - 1; i++) {
    int min = i;
    for (int j = i + 1; j < n; j++) {
      if (a[j] <= a[min]) min = j; 
    }
    int t = a[min]; a[min] = a[i]; a[i] = t; 
  }
}
\end{minted}
\subsection*{Selection Sort mit IndexOf}
\begin{minted}{java}
void sort(int[] data) {
  int n = data.length;
  for (int i = 0; i < n - 1; i++) {
    int min = indexOfMin(data, i, n);
    int t = data[min];
    data[min] = data[i];
    data[i] = t;
  }
}
\end{minted}